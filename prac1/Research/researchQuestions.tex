\documentclass[a4paper]{article}

%\usepackage[margin=2cm]{geometry}
%\usepackage{qtree}
%\usepackage{color}
%\usepackage{forest}
%\usepackage{listings}
\usepackage{tikz}	%for the graphics
\usepackage{cite} 	%for bibtex
\usepackage{url}	%for the url's in bibtex



\begin{document}
\begin{titlepage}
	\begin{center}
		\begin{figure}[t]
			\centering
			\includegraphics[width=350px]{logo.PNG}
		\end{figure}
		
		\begin{center}
			\textsc{\Huge Programming Languages}
		\end{center}
		\begin{center}
			\textsc{\Huge COS 333}
		\end{center}
		\begin{center}		
			\textsc{\LARGE Practical Lab Experience 1:}		
		\end{center}
		\begin{center}		
			\textsc{\LARGE Research Assignment}		
		\end{center}
		
		\begin{flushright} \large
			Juan Jaques du Preez \newline \emph{u15189016} \newline
		\end{flushright}
\par\vspace{\fill}
{\large Date:}
\\
{\large \today}

	\end{center}
\end{titlepage}

\tableofcontents
\newpage

\section{Question 1: Esolang}
According to esolang \cite{esolang}, an esoteric programming language, or esolang, is a computer programming language that is designed to experiment with peculiar ideas and to be a joke, rather than for practical use. The word \textit{esoteric} means "likely to be understood by only a small number of people\cite{wikipedia}." This implies that such a language would only be understood by a small number of developers. 
\section{Question 2: Views on Esoteric languages}
	\subsection{for}
	potentials
		creative thinking
		expression
		reach new ways of programming that may aid in future
	expression of an art
	tests the boundaries of computer science
	
	\subsection{against}
	list some really silly ones
	seems like waste of time
	unstructured means informal means one could have spent the time doing something more focused
	list costs
\section{Question 3: Practical Examples}
	\subsection{esolang one}
	name
	description
		designer
		year of initial design
		syntax
		semantics
		Turing complete
		
	short code snipped
	short explanation of code snippet
	\subsection{esolang two}
	name
	description
		designer
		year of initial design
		syntax
		semantics
		Turing complete
		
	short code snipped
	short explanation of code snippet
\section{Question 4: Stack-based Programming Languages}
	brief explanation of what it is
	short example
\section{Question 5: Turing (the programming language)}
	good introductory
	not good instructional
\section{Question 6: Design by Contract}
	what it is
	list two languages
\section{Question 7: AWK Programming Language}
	intended application area
	overall syntactic structure
	explain how these two work together

\bibliography{mybib}{}
\bibliographystyle{plain}
	
\end{document}
