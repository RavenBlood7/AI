\documentclass[a4paper]{article}

%\usepackage[margin=2cm]{geometry}
%\usepackage{qtree}
%\usepackage{color}
%\usepackage{forest}
\usepackage{listings}
\usepackage{tikz}	%for the graphics
\usepackage{cite} 	%for bibtex
\usepackage{url}	%for the url's in bibtex



\begin{document}
\begin{titlepage}
	\begin{center}
		\begin{figure}[t]
			\centering
			\includegraphics[width=350px]{logo.PNG}
		\end{figure}
		
		\begin{center}
			\textsc{\Huge Programming Languages}
		\end{center}
		\begin{center}
			\textsc{\Huge COS 333}
		\end{center}
		\begin{center}		
			\textsc{\LARGE Practical Lab Experience 1:}		
		\end{center}
		\begin{center}		
			\textsc{\LARGE Research Assignment}		
		\end{center}
		
		\begin{flushright} \large
			Juan Jaques du Preez \newline \emph{u15189016} \newline
		\end{flushright}
\par\vspace{\fill}
{\large Date:}
\\
{\large \today}

	\end{center}
\end{titlepage}

\tableofcontents
\newpage

\section{Question 1: Esolang}
According to esolang \cite{esolang}, an esoteric programming language, or esolang, is a computer programming language that is designed to experiment with peculiar ideas and to be a joke, rather than for practical use. The word \textit{esoteric} means "likely to be understood by only a small number of people\cite{wikipedia}." This implies that such a language would only be understood by a small number of developers. 
\section{Question 2: Views on Esoteric languages}
	\subsection{against}
	potentials
		lateral thinking
		creative thinking
		expression
		reach new ways of programming that may aid in future
	expression of an art
	tests the boundaries of computer science
	
	Although esoteric languages have been considered as no more than jokes, they do seem to carry some potential. This potential may not be to write practical programs, or even to be used in the real world, but there are other possibilities to consider when thinking about these languages. First of all, it stimulates creative and lateral thinking. \cite{wikipedia2} explains that \textit{lateral thinking} means "an indirect method of thinking which is not immediately obvious and involves ideas which may not be obtainable by using the traditional step-by-step method of thinking." This "out of the box" means of thought introduces brand new ideas of solving problems, which may turn out useful in computer science research. These problem solving methods could be a big aid to test the boundaries and limits of our current knowledge of computer science, as well as information science.
\newline
\newline	
	Also, it can be considered as an expression in the form of art\cite{esolangArt}. Some of these languages are no more useful than a painting hanging in a gallary. We made such a difficult language simply because we can, in the same way a water-colour painting is a representation of the artist's abilities and talents. The programming language of Piet is a good example of this, as it transforms each program into a literal work of modern art. (see figure 1)
	
	\subsection{for}
	list some really silly ones
	seems like waste of time
	unstructured means informal means one could have spent the time doing something more focused
	list costs
	
	However, it must be said that some esoteric languages are specifically just for fun. For example, the language LOLCODE is based off the LOLCats phenomenon\cite{esolangArt}. It uses silly words and phrases that would commonly be used by LOLCats to write programs. A lot of this could be seen as a waste of time. These programmers could have spent their time doing formal research into some computer science prinicples. One could have seen a better furthering of research if their time spent were more focused. 
	
			\begin{figure}[h]
			\centering
			\includegraphics[width=200px]{helloworld-piet.png}
			\caption{Piet Hello World Program}
		\end{figure}
\section{Question 3: Practical Examples}
	\subsection{Esolang one: Ook!}
	name
	description
		designer
		year of initial design
		syntax
		semantics
		Turing complete
		
	short code snipped
	short explanation of code snippet
	
	Ook! is an esoteric language based on another esoteric language called BrainF***. It was created by a man called David Morgan-Mar even before the year 2008. Both of these languages are Turing-complete\cite{ook2}. There are only three distinct syntax elements:
	\begin{itemize}
	\item Ook.
	\item Ook?
	\item Ook!
	\end{itemize}
	These three elements are combined to form the eight BrainF*** commands: \cite{ook}
\begin{itemize}

    \item Ook. Ook?\\
    Move the Memory Pointer to the next array cell.
    \item Ook? Ook.\\
    Move the Memory Pointer to the previous array cell.
    \item Ook. Ook.\\
    Increment the array cell pointed at by the Memory Pointer.
    \item Ook! Ook!\\
    Decrement the array cell pointed at by the Memory Pointer.
    \item Ook. Ook!\\
    Read a character from STDIN and put its ASCII value into the cell pointed at by the Memory Pointer.
    \item Ook! Ook.\\
    Print the character with ASCII value equal to the value in the cell pointed at by the Memory Pointer.
    \item Ook! Ook?\\
    Move to the command following the matching Ook? Ook! if the value in the cell pointed at by the Memory Pointer is zero. Note that Ook! Ook? and Ook? Ook! commands nest like pairs of parentheses, and matching pairs are defined in the same way as for parentheses.
    \item Ook? Ook!\\
    Move to the command following the matching Ook! Ook? if the value in the cell pointed at by the Memory Pointer is non-zero. 
\end{itemize}	
	The code lacks a lot in the readability and writability departments, as there are only three different possibilities of lexical tokens.
	\begin{lstlisting}
Ook. Ook? Ook. Ook. Ook. Ook. Ook. Ook. Ook. Ook. Ook. 
Ook. Ook. Ook. Ook. Ook. Ook. Ook. Ook. Ook. Ook! Ook?
Ook? Ook. Ook. Ook. Ook. Ook. Ook. Ook. Ook. Ook. Ook. 
Ook. Ook. Ook. Ook. Ook. Ook. Ook. Ook. Ook? Ook! Ook! 
Ook? Ook! Ook? Ook. Ook! Ook. Ook. Ook? Ook. Ook. Ook. 
Ook. Ook. Ook. Ook. Ook. Ook. Ook. Ook. Ook. Ook. Ook. 
Ook! Ook? Ook? Ook. Ook. Ook. Ook. Ook. Ook. Ook. Ook. 
Ook. Ook. Ook? Ook! Ook! Ook? Ook! Ook? Ook. Ook. Ook. 
Ook! Ook. Ook. Ook. Ook. Ook. Ook. Ook. Ook. Ook. Ook. 
Ook. Ook. Ook. Ook. Ook. Ook! Ook. Ook! Ook. Ook. Ook. 
Ook. Ook. Ook. Ook. Ook! Ook. Ook. Ook? Ook. Ook? Ook. 
Ook? Ook. Ook. Ook. Ook. Ook. Ook. Ook. Ook. Ook. Ook. 
Ook. Ook. Ook. Ook. Ook. Ook. Ook! Ook? Ook? Ook. Ook. 
Ook. Ook. Ook. Ook. Ook. Ook. Ook. Ook. Ook? Ook! Ook! 
Ook? Ook! Ook? Ook. Ook! Ook. Ook. Ook? Ook. Ook? Ook. 
Ook? Ook. Ook. Ook. Ook. Ook. Ook. Ook. Ook. Ook. Ook.
Ook. Ook. Ook. Ook. Ook. Ook. Ook. Ook. Ook. Ook. Ook! 
Ook? Ook? Ook. Ook. Ook. Ook. Ook. Ook. Ook. Ook. Ook. 
Ook. Ook. Ook. Ook. Ook. Ook. Ook. Ook. Ook. Ook. Ook. 
Ook? Ook! Ook! Ook? Ook! Ook? Ook. Ook! Ook! Ook! Ook! 
Ook! Ook! Ook! Ook. Ook? Ook. Ook? Ook. Ook? Ook. Ook? 
Ook. Ook! Ook. Ook. Ook. Ook. Ook. Ook. Ook. Ook! Ook. 
Ook! Ook! Ook! Ook! Ook! Ook! Ook! Ook! Ook! Ook! Ook! 
Ook! Ook! Ook. Ook! Ook! Ook! Ook! Ook! Ook! Ook! Ook! 
Ook! Ook! Ook! Ook! Ook! Ook! Ook! Ook! Ook! Ook. Ook. 
Ook? Ook. Ook? Ook. Ook. Ook! Ook. 
	\end{lstlisting}
	\subsection{Esolang two: Whenever}
	name
	description
		designer
		year of initial design
		syntax
		semantics
		Turing complete
		
	short code snipped
	short explanation of code snippet
	
	Whenever is another interesting programming language created by David Morgan-Mar. The language is said to be turing complete. The idea behind Whenever is that the program does a list of tasks in whichever order it wants to. It definitely gets to each task, but does it whenever it feels like it. The syntax works as follows. a single item on the todo list is constructed by a line number followed by a statement. Statements can be forgotten, deferred, or done more than once. This is done with the keywords forget, defer, and again. These keywords specify whether an item should be removed from the todo list, remain on the todo list after executing the line, or be done later.
	
	Here is an example of printing out the first 100 fibonacci numbers:
	\begin{lstlisting}
1 again (1) defer (3 || N(1)<=N(2) || N(7)>99) 2#N(1),3,7;
2 again (2) defer (3 || N(2)<=N(1) || N(7)>99) 1#N(2),3,7;
3 defer (5) print(N(1)+N(2));
4 defer (5) print("1");
5 4,-3,7;
6 defer (4) 3;
7 7;
8 defer (N(7)<100) -1#N(1),-2#N(2),-7#100,-3;
9 defer (3 || 6) 1,3; 
	\end{lstlisting}
\section{Question 4: Stack-based Programming Languages}
	brief explanation of what it is
	short example
	
	A stack-based programming language is a language which makes use of an implicit stack or several stacks as a fundemental design for the language. It uses the pushing and popping qualities of the stack to pass parameters between certain operations and functions. 
	Take for example the programming language PostScript. As its name suggests, it uses post-fix notation to perform operations. To explain this, I will use the example of a function add(4,5). Infix notation is "4 + 5", but postfix notation is "4 5 +." In a stack based programming language, the function would be "4 5 add." The programming language pushes 4 and 5 onto the stack. When "add" is read, the two numbers are popped off the stack, added together, and the result is pushed onto the stack again. So, the parameters, as well as the return values, for the "add" function are accessed from a single stack. These characteristics of the stack is used throughout the program, and among all the other functions.
\section{Question 5: Turing (the programming language)}
	good introductory
	not good instructional
	
	From the special keywords and syntax, it is clear that Turing was designed to be very readable. The	use of words such as "begin," "end," "var," and "loop" suggest that it is imperative that the user understands what the program is doing at all times. It is good for teaching programming, because it portrays core programming principles in easy-to-understand pseudocode which is designed to be close to natural language. Thus, the focus is on teaching the principles, rather than the syntax of the language. A fully featured programming language like C++ or java has many syntactic issues to remember, which ought not to be the focus when just starting with programming.
	
	However, as soon as one is used to the concepts of programming, one no longer has need for such an introductory language as Turing. It becomes redundant to write out "begin," "end," or "end if" every time when a single symbol would suffice. If you understand when a selection statement or loop begins and ends, and are used to the logic behind the syntax, then one might as well use the short-hand syntax.
	
	
\section{Question 6: Design by Contract}
	what it is
	list two languages
	
	According to \url{wiki.c2.com} \cite{c2}, Design by Contract is "a software correctness methodology that uses preconditions and postconditions to document (or programmatically assert) the change in state caused by a piece of the program." In essence, formal interface specifications for software components are created in the form of "contracts," which contain these specifications of design. Quality control is a key element of Design by Contract. One achieves a certain quantifiable level of software quality by specifying obligations and expectations for the different components in the system.
	
	Eiffel (created in 1986) is one example of a porgramming language that conforms to the Design by Contract methodology. Eiffel was designed specifcally for the purpose of providing a reliable, formal language which can be used in industry, so the design methodology was perfect for this purpose. 
	
	Another example that natively supports DbC is Cobra, a general pupose object-oriented programming language.
	
\section{Question 7: AWK Programming Language}
	intended application area
	overall syntactic structure
	explain how these two work together

\bibliography{mybib}{}
\bibliographystyle{plain}
	
\end{document}
