\documentclass[a4paper]{article}

\usepackage[margin=0.5cm]{geometry}
\usepackage{qtree}
\usepackage{color}
\usepackage{forest}
\usepackage{tikz}
\usepackage{listings}

\begin{document}
\begin{titlepage}
	\begin{center}
		\begin{figure}[t]
			\centering
			\includegraphics[width=350px]{logo.PNG}
		\end{figure}
		
		\begin{center}
			\textsc{\Huge Programming Languages}
		\end{center}
		\begin{center}
			\textsc{\Huge COS 333}
		\end{center}
		\begin{center}		
			\textsc{\LARGE Practical Lab Experience 1:}		
		\end{center}
		\begin{center}		
			\textsc{\LARGE Research Assignment}		
		\end{center}
		
		\begin{flushright} \large
			Juan Jaques du Preez \newline \emph{u15189016} \newline
		\end{flushright}
\par\vspace{\fill}
{\large Date:}
\\
{\large \today}

	\end{center}
\end{titlepage}

\tableofcontents
\newpage

\section{Overview}
	\subsection{Overall UML Diagram}
		\begin{center}
			\includegraphics[width=350px]{overview.jpg}	
		\end{center}
		
	\subsection{Options Implemented}
	\begin{itemize}
	\item -d
	\item -c
	\item -md
	\item -mc
	\item -pc
	\item -pd
	\end{itemize}		
	
	\subsection{Please Note}
	The code is not as efficient as it could be. For example, I have two functions in DiscreteDT called induceWithMissing() and induceNoMissing(). These two are very much the same, but I wanted to make the different algorithms that were used as clear as possible.
\section{Compile and Run Program}
For this project I used the language of C++.
	\subsection{Compile}
	To compile the program, open a terminal and type in the following commands:
	\newline
	\newline
	\textbf{cd DecisionTree/} \newline
	\textbf{make} 
	\newline
	\newline
	
	\subsection{Run}
	To run the program, make sure you are in the "DecisionTree/" directory. Copy the two files (data file and spec file) into 			this directory. Open a terminal and type: \newline
	\newline	
	\textbf{./DecisionTree  option specfile datafile}
	\newline	
	\newline
	Where option is one of: -d, -c, -md, -mc, -pd, -pc.
	\newline
	\newline
	While inducing a tree and pruning a tree, there will be some values displayed on the screen. This is just to make sure that 		the program is doing something. When it is finished, the last part will be the requested output. The requested output will 			also be written to a file called "data.out".
	\newline	
	\newline
	The game also keeps track of how many seeds are in the stack of each player, as well as how many seeds are currently in the 			hand (the value on the right side of the board).
	\newline	
	\newline
	When an AI is playing, the alphaBeta pruning functionality can be followed in the terminal. It gives exactly which states it 	considers and which it prunes based on what knowledge.
	
	
	\subsection{Contact}
	If you cannot get the program to compile and run, or have any other issue with regards to anything, please contact me: \newline
	Juan du Preez\newline
	078 141 0915\newline
	u15189016@tuks.co.za\newline
	juan.dupreez82@gmail.com
	
	\subsection{What does not work}
	To make your job easier, here is a list of things that do not work: \newline
	
	
\end{document}
